\documentclass[a4paper]{IEEEtran}
\usepackage[utf8]{inputenc}
\usepackage[T1]{fontenc}
\usepackage{amsfonts, amsthm,amsmath}
\usepackage{graphicx}

\title{Civilizacije}
\author{Anže Marinko \\ mentor: dr. Matjaž Gams \\ Inštitut Jožef Stefan - Ljubljana \\ pomlad 2020}

\begin{document}

\maketitle

\section{Uvod, malo o kodi, metodah}
We want to estimate distribution of time that some civilisation
survives in our galaxy.

All code is made in Python3.

We are generating data by different models ... Each model takes two more parameters:
* **maxN** ... maximal number of civilisations in our galaxy. We assume
that there is at least one civilisation (here we are) and at most 10 000
civilisations otherwise we would already see them.
* **distribution** ... some parameters of the model are only bounded
and we have to define distributions in this bounds. Possible distributions are:
"loguniform", "uniform", "halfgauss", "lognormal", "fixed".

Models

* **Model 1:** The most elementary model using Sandberg distribution explained
in `Sandberg-original-paper.pdf`,
* **Model 2:** We collect together some variables from first model so
we have only two random variables,
* **Model 3:** We add a possibility of spreading to other
planets to first model.

Save collected data for further analysis

`invalid-parameters.txt` is list of parameters that gave invalid input data
(more than 5 \% of probability grater than 10\^15 years)

Moments are central mathematical moments. First moment is mean and
second moment is variance. We can make histograms in linear or
logarithmic scale.

Data comparison: Drawing of selected distributions

We can draw generated data using `plot3D\_L.py` where
we have to set scale (logarithmic scale or not) and models
to draw.

Clustering of generated distributions

We can cluster al histograms in selected number of clusters
using `cluster.py`. It draws us some plots so we can understand
generated data better. In cloud of points colored by model
size of marker is equal to `log(maxN)`.

* scale of *x* axis for histograms `logarithmic\_scale`
* list of numbers of clusters to make `ks` from 1 to 10 so it
makes an analysis of data clustered ib each number of clusters
from the list
* `by\_histograms` should be *True* to cluster by histogram
space

\section{Rezultati, opažanja, predvidevanja ob slikah, predlogi za naprej}

\end{document}
